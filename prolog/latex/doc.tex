% This LaTeX document was generated using the LaTeX backend of PlDoc,
% The SWI-Prolog documentation system



\subsection{cli.pl}

\label{sec:cli}

\begin{description}
    \predicate{cli}{0}{}
The \predref{cli}{0} predicate is the main entry point for the command-line interface.
It reads a line of user input, determines the corresponding output, and prints the output.
If the input is `stop', the predicate terminates.
Otherwise, it calls itself to read the next line of input.

    \predicate[private]{handle_input}{2}{+Input:string, -Output:string}
The \predref{handle_input}{2} predicate preprocesses the input and tries to parse it as a sentence, question, or command.
If the input cannot be parsed, it outputs a default response.

\begin{arguments}
\arg{\Splus} & \arg{Input} The user input. \\
\arg{\Sminus} & \arg{Output} The generated output. \\
\end{arguments}

    \predicate[private]{preprocess_input}{2}{+Input:string, -Output:list}
The \predref{preprocess_input}{2} predicate splits the input on whitespace into a list of words.
It transforms each word to a lowercase atom.

\begin{arguments}
\arg{\Splus} & \arg{Input} The user input. \\
\arg{\Sminus} & \arg{Output} The list of atoms. \\
\end{arguments}

    \predicate[private]{handle_sentence}{2}{+Sentence:list, -Output:string}
The \predref{handle_sentence}{2} predicate handles a sentence input.
If the fact that corresponds to the sentence is not known, it is added to the known facts.

\begin{arguments}
\arg{\Splus} & \arg{Sentence} The sentence (a list of atoms). \\
\arg{\Sminus} & \arg{Output} The generated output. \\
\end{arguments}

    \predicate[private]{handle_question}{2}{+Question:list, -Output:string}
The \predref{handle_question}{2} predicate handles a question input.
It tries to prove the question and/or its negation with the question-answering engine.

\begin{arguments}
\arg{\Splus} & \arg{Question} The question (a list of atoms). \\
\arg{\Sminus} & \arg{Output} The generated output. \\
\end{arguments}
\end{description}

\subsection{command.pl}

\label{sec:command}

\begin{description}
    \dcg[private]{command}{3}{-Goal:atom, -Output:string, +Words:list}
The \dcgref{command}{3} DCG rule parses a list of atoms into a goal.
It executes the goal with the engine module, outputs the response.

\begin{arguments}
\arg{\Sminus} & \arg{Goal} The goal to execute. \\
\arg{\Sminus} & \arg{Output} The generated output. \\
\arg{\Splus} & \arg{Words} The command (a list of atoms). \\
\end{arguments}
\end{description}

\subsection{engine.pl}

\label{sec:engine}

\begin{description}
    \predicate{prove_question}{2}{+Question:atom, -Output:string}
The \predref{prove_question}{2} predicate tries to prove a question from the known facts.
If the question can be proved either way, it outputs the answer.
Otherwise, it outputs a default response.

\begin{arguments}
\arg{\Splus} & \arg{Question}: The question to prove. \\
\arg{\Sminus} & \arg{Output}: The generated output.
  \\
\end{arguments}

    \predicate{prove_question_list}{2}{+Question:atom, -Output:string}
The \predref{prove_question_list}{2} predicate tries to prove a question from the known facts.
If the question can be proved either way, it outputs the answer.
Otherwise, it outputs the empty string.
It is suitable for use with maplist, such as in \predref{find_known_facts_noun}{2}.

\begin{arguments}
\arg{\Splus} & \arg{Question}: The question to prove. \\
\arg{\Sminus} & \arg{Output}: The generated output.
  \\
\end{arguments}

    \predicate{prove_question_tree}{2}{+Question:atom, -Output:string}
The \predref{prove_question_tree}{2} predicate is an extended version of \predref{prove_question}{2} that constructs a proof tree.
If the question can be proved either way, it transforms each step of the proof into a
sentence and concatenates the sentences into the output.

\begin{arguments}
\arg{\Splus} & \arg{Question}: The question to prove. \\
\arg{\Sminus} & \arg{Output}: The generated output.
  \\
\end{arguments}

    \predicate[private]{prove_from_known_facts}{6}{+Clause:atom, +TruthValue:atom, -Certainty:int, +FactList:list, -ProofList:list, -Proof:atom}
The \predref{prove_from_known_facts}{4} predicate tries to prove a clause based on a list of facts.
If the clause can be proved, it stores the proof in the output.
The proof is a list of steps, where each step is a fact that was used to prove the clause.

\begin{arguments}
\arg{\Splus} & \arg{Clause}: The clause to prove. \\
\arg{\Splus} & \arg{TruthValue}: The truth value of the clause (true or false). \\
\arg{\Sminus} & \arg{Certainty}: The accumulated certainty of the proof. \\
\arg{\Splus} & \arg{FactList}: The list of facts to use. \\
\arg{\Splus} & \arg{ProofList}: The accumulator for the proof. \\
\arg{\Sminus} & \arg{Proof}: The generated proof.
  \\
\end{arguments}

    \predicate[private]{find_known_facts}{1}{-Output:string}
The \predref{find_known_facts}{1} predicate finds all known facts and transforms the sentences
into a string output.

\begin{arguments}
\arg{\Sminus} & \arg{Output}: The known facts or default response.
  \\
\end{arguments}

    \predicate[private]{find_known_facts_noun}{2}{+ProperNoun:atom, -Output:string}
The \predref{find_known_facts_noun}{2} predicate finds all known facts with a proper-noun argument,
and transforms the sentences into a string output.

\begin{arguments}
\arg{\Splus} & \arg{ProperNoun}: The proper noun. \\
\arg{\Sminus} & \arg{Output}: The known facts or default response.
  \\
\end{arguments}

    \predicate[private]{is_fact_known}{1}{+FactList:list}
The \predref{is_fact_known}{2} predicate checks if a fact is known.
If the fact itself is not known, it tries to prove it based on the known facts.

\begin{arguments}
\arg{\Splus} & \arg{FactList}: The list of facts to check.
  \\
\end{arguments}

    \predicate[private]{add_clause_to_facts}{3}{+Clause:atom, +FactListOld:list, -FactListNew:list}
The \predref{add_clause_to_facts}{3} predicate adds a clause to a list of facts.

\begin{arguments}
\arg{\Splus} & \arg{Clause}: The clause. \\
\arg{\Splus} & \arg{FactListOld}: The list of facts to update. \\
\arg{\Sminus} & \arg{FactListNew}: The updated list of facts.
  \\
\end{arguments}

    \predicate[private]{output_answer}{2}{+Question:atom, -Output:string}
The \predref{output_answer}{2} predicate transforms a question answer into a string output.

\begin{arguments}
\arg{\Splus} & Result: The question. \\
\arg{\Sminus} & \arg{Output}: The generated output.
  \\
\end{arguments}

    \predicate[private]{output_known_fact}{2}{+Fact:atom, -Output:string}
The \predref{output_known_fact}{2} predicate transforms a known fact into a string output.

\begin{arguments}
\arg{\Splus} & \arg{Fact}: The known fact. \\
\arg{\Sminus} & \arg{Output}: The generated output.
  \\
\end{arguments}

    \predicate[private]{output_proof}{2}{+Proof:atom, -Output:string}
The \predref{output_proof}{2} predicate transforms a proof step into a string output.

\begin{arguments}
\arg{\Splus} & \arg{Proof}: The proof step. \\
\arg{\Sminus} & \arg{Output}: The generated output.
  \\
\end{arguments}

    \predicate[private]{output_proof_list}{3}{+Question:atom, +ProofList:list, -Output:string}
The \predref{output_proof_list}{2} predicate transforms a question and a proof list into a string output.

\begin{arguments}
\arg{\Splus} & \arg{Question}: The question. \\
\arg{\Splus} & \arg{ProofList}: The proof list. \\
\arg{\Sminus} & \arg{Output}: The generated output.
  \\
\end{arguments}
\end{description}

\subsection{grammar.pl}

\label{sec:grammar}

\begin{description}
    \predicate{predicate}{3}{-Predicate:atom, -Arity:integer, -Words:list}
The \predref{predicate}{3} predicate defines the logical vocabulary of the system.
It relates a predicate, its arity, and a list of words and syntactic categories that refer to the logical entity.

\begin{arguments}
\arg{Predicate} & The predicate. \\
\arg{Arity} & The number of arguments the predicate takes. \\
\arg{Words} & A list of words that refer to the logical entity.
  \\
\end{arguments}

    \dcg[private]{proper_noun}{2}{?Number:atom, ?Word:atom}
The \dcgref{proper_noun}{2} DCG rule defines proper nouns.
It relates the proper noun's grammatical number, its atom, and a list of atoms that refer to it.

\begin{arguments}
\arg{Number} & The grammatical number. \\
\arg{Word} & The proper noun.
  \\
\end{arguments}

    \dcg[private]{verb_phrase}{2}{?Number:atom, ?Word:atom}
The \dcgref{verb_phrase}{2} DCG rule defines verb phrases.
It relates the verb phrase's grammatical number, its atom, and a list of atoms that refer to it.

\begin{arguments}
\arg{Number} & The grammatical number. \\
\arg{Word} & The property or verb.
  \\
\end{arguments}

    \dcg[private]{property}{2}{?Number:atom, ?Word:atom}
The \dcgref{property}{2} DCG rule defines properties.
It relates the property's grammatical number, its atom, and a list of atoms that refer to it.

\begin{arguments}
\arg{Number} & The grammatical number. \\
\arg{Word} & The adjective or noun.
  \\
\end{arguments}

    \dcg[private]{determiner}{4}{?Number:atom, ?Body:atom, ?Head:atom, ?Rule:atom}
The \dcgref{determiner}{4} DCG rule defines determiners.
It relates the determiner's grammatical number, its corresponding rule, and a list of atoms that refer to it.

\begin{arguments}
\arg{\Squest} & \arg{Number} The grammatical number. \\
\arg{\Squest} & \arg{Body} The relation between X and the body of the rule. \\
\arg{\Squest} & \arg{Head} The relation between X and the head of the rule. \\
\arg{\Squest} & \arg{Rule} The rule.
  \\
\end{arguments}

    \dcg[private]{adjective}{1}{?ToLiteral:atom}
The \dcgref{adjective}{1} DCG rule defines adjectives.
It relates the adjective and its literal.

\begin{arguments}
\arg{\Squest} & \arg{ToLiteral} The adjective and its literal in the form `Noun \Sssu{} Adjective(Noun)`.
  \\
\end{arguments}

    \dcg[private]{noun}{2}{?Number:atom, ?ToLiteral:atom}
The \dcgref{noun}{2} DCG rule defines common nouns.
It relates the noun, its grammatical number, and its literal.

\begin{arguments}
\arg{\Squest} & \arg{Number} The grammatical number. \\
\arg{\Squest} & \arg{ToLiteral} The noun and its literal.
  \\
\end{arguments}

    \dcg[private]{intransitive_verb}{2}{?Number:atom, ?ToLiteral:atom}
The \dcgref{intransitive_verb}{2} DCG rule defines intransitive verbs.
It relates the verb, its grammatical number, and its literal.

\begin{arguments}
\arg{\Squest} & \arg{Number} The grammatical number. \\
\arg{\Squest} & \arg{ToLiteral} The intransitive verb and its literal.
  \\
\end{arguments}

    \predicate{noun_singular_to_plural}{2}{?SingularNoun:atom, ?PluralNoun:atom}
The \predref{noun_singular_to_plural}{2} predicate converts the singular form of a noun to the
plural form and vice versa.

\begin{arguments}
\arg{SingularNoun} & The singular form. \\
\arg{PluralNoun} & The plural form.
  \\
\end{arguments}

    \predicate{verb_plural_to_singular}{2}{?PluralVerb:atom, ?SingularVerb:atom}
The \predref{verb_plural_to_singular}{2} predicate converts the plural form of a verb to the
singular form and vice versa.

\begin{arguments}
\arg{PluralVerb} & The plural form. \\
\arg{SingularVerb} & The singular form.
  \\
\end{arguments}

    \predicate{predicate_to_grammar}{4}{+Predicate:atom, +Arity:integer, +WordCategory:atom, -ToLiteral:atom}
The \predref{predicate_to_grammar}{4} predicate constructs a literal from a predicate and an argument.

\begin{arguments}
\arg{Predicate} & The predicate. \\
\arg{Arity} & The number of predicate arguments. \\
\arg{WordCategory} & The word category. \\
\arg{ToLiteral} & The word and its literal.
  \\
\end{arguments}
\end{description}

\subsection{question.pl}

\label{sec:question}

\begin{description}
    \dcg{question}{1}{?Question:list}
The \dcgref{question}{1} DCG rule parses a list of atoms into a question.

\begin{arguments}
\arg{Question} & The list of atoms. \\
\end{arguments}

    \dcg[private]{question_body}{1}{?Question:list}
The \dcgref{question_body}{1} DCG rule parses a list of atoms into the body of a question.

\begin{arguments}
\arg{Question} & The list of atoms. \\
\end{arguments}
\end{description}

\subsection{sentence.pl}

\label{sec:sentence}

\begin{description}
    \dcg{sentence}{1}{?Sentence:list}
The \dcgref{sentence}{1} DCG rule parses a list of atoms into a sentence.

\begin{arguments}
\arg{Sentence} & The list of atoms. \\
\end{arguments}

    \dcg{sentence_body}{1}{?Sentence:list}
The \dcgref{sentence_body}{1} DCG rule parses a list of atoms into a sentence body.
The operator \verb$=>$ is used to unify two atoms in \verb$grammar.pl$, usually ToLiteral.
This is handy because we need to pass them both through several layers of the grammar module,
e.g., \dcgref{verb_phrase}{2} to \dcgref{property}{2} to \dcgref{adjective}{1} to \predref{predicate_to_grammar}{4}.
The operator itself does not do anything special -- it could be any arithmetic operator,
provided that it is destructured accordingly in \predref{predicate_to_grammar}{4}.

\begin{arguments}
\arg{Sentence} & The list of atoms. \\
\end{arguments}
\end{description}

\subsection{utils.pl}

\label{sec:utils}

\begin{description}
    \predicate{concatenate_conjunctive}{3}{+ListX:list, +ListY:list, -ListZ:list}
The \predref{concatenate_conjunctive}{3} predicate concatenates two lists of literals.

\begin{arguments}
\arg{ListX} & The first list. \\
\arg{ListY} & The second list. \\
\arg{ListZ} & The concatenated list. \\
\end{arguments}

    \predicate[private]{find_clause}{3}{+Clause:atom, +Fact:atom, +FactList:list}
The \predref{find_clause}{3} predicate finds a clause in the list of facts that unifies with the
given clause and outputs the fact (without instantiating it).

\begin{arguments}
\arg{\Splus} & \arg{Clause}: The clause to find. \\
\arg{\Splus} & \arg{Fact}: The output fact. \\
\arg{\Splus} & \arg{FactList}: The list of facts to search. \\
\end{arguments}

    \predicate[private]{transform}{2}{+Term:atom, -ClauseList:list}
The \predref{transform}{2} predicate transforms a term into a list of clauses.

\begin{arguments}
\arg{\Splus} & \arg{Term}: The term to transform. \\
\arg{\Sminus} & \arg{ClauseList}: The list of clauses generated based on the term. \\
\end{arguments}

    \predicate{try}{1}{+X}
The \predref{try}{1} predicate tries to prove a goal.
The double use of the negation-as-failure oeprator undoes bindings of variables.

\begin{arguments}
\arg{X} & The goal to prove. \\
\end{arguments}

    \predicate[private]{write_debugs}{1}{+ListX:list}
The \predref{write_debugs}{1} predicate writes a list of debug messages to the console.

\begin{arguments}
\arg{ListX} & : A list of debug messages. \\
\end{arguments}

    \predicate[private]{write_debug}{1}{+X}
The \predref{write_debug}{1} predicate writes a debug message to the console.

\begin{arguments}
\arg{X} & : The debug message. \\
\end{arguments}
\end{description}

